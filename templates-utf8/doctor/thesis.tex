\documentclass[%
doctor,      % тип документа
%natbib,      % использовать пакет natbib для "сжатия" цитирований
subf,        % использовать пакет subcaption для вложенной нумерации рисунков
href,        % использовать пакет hyperref для создания гиперссылок
colorlinks,  % цветные гиперссылки
%fixint,     % включить прямые знаки интегралов
%classified, % гриф секретности
%facsimile,  % отображать факсимиле диссертанта
]{disser}

\usepackage[
  a4paper, mag=1000,
  left=2.5cm, right=1cm, top=2cm, bottom=2cm, headsep=0.7cm, footskip=1cm
]{geometry}

\usepackage[intlimits]{amsmath}
\usepackage{amssymb,amsfonts}

\usepackage[T2A]{fontenc}
\usepackage[utf8]{inputenc}
\usepackage[english,russian]{babel}
\ifpdf\usepackage{epstopdf}\fi
\usepackage[autostyle]{csquotes}

% Список сокращений и условных обозначений
\usepackage[intoc,nocfg,russian]{nomencl}
\newcommand{\nomencl}[2]{#1 --- #2\nomenclature{#1}{#2}}
\setlength{\nomlabelwidth}{3em}
\setlength{\nomitemsep}{-\parsep}
\renewcommand{\nomlabel}[1]{#1 ---}
\makenomenclature

% Шрифт Times в тексте как основной
%\usepackage{tempora}
% альтернативный пакет из дистрибутива TeX Live
%\usepackage{cyrtimes}

% Шрифт Times в формулах как основной
%\usepackage[varg,cmbraces,cmintegrals]{newtxmath}
% альтернативный пакет
%\usepackage[subscriptcorrection,nofontinfo]{mtpro2}

% Плавающие рисунки "в оборку".
\usepackage{wrapfig}

\usepackage[%
  style=gost-numeric,
  backend=biber,
  language=auto,
  hyperref=auto,
  autolang=other,
  sorting=none
]{biblatex}

% Настройки biblatex
%% символ номера
\DefineBibliographyStrings{russian}{number={\textnumero}}
%% запятая между номерами цитирований
\renewcommand*{\multicitedelim}{,\space}
%% короткое тире в диапазоне цитирований
\DefineBibliographyExtras{russian}{\renewcommand*{\bibrangedash}{--}}

\addbibresource{thesis.bib}

% Номера страниц снизу и по центру
%\pagestyle{footcenter}
%\chapterpagestyle{footcenter}

% Ссылки на работы соискателя включаются в общий список литературы
\let\citeown=\cite

% Использовать полужирное начертание для векторов
\let\vec=\mathbf

% Путь к файлам с иллюстрациями
\graphicspath{{fig/}}

\begin{document}

% Переопределение стандартных заголовков
%\def\contentsname{Содержание}
%\def\conclusionname{Выводы}
%\def\bibname{Литература}
%\def\listfigurename{Список иллюстраций}

% Включение файла с общим текстом диссертации и автореферата
% (текст титульного листа и характеристика работы).
\input{common}

% номер копии для грифа секретности
%\copynum{1}
% класс доступа
%\classlabel{Для служебного пользования}

% номер УДК
\libcatnum{12345}

\title{ДИССЕРТАЦИЯ\\
на соискание ученой степени\\
доктора физико-математических наук}

\maketitle

%%
%% Titlepage in English
%%
%
%\institution{Name of Organization}
%
%\title{Doctoral Dissertation}
%
%% Topic
%\topic{Dummy Title}
%
%% Author
%\author{Author's Name}
%
%\specnum{01.04.05}
%\spec{Optics}
%
%%\specsndnum{01.04.07}
%%\specsnd{Condensed matter physics}
%
%% Scientific consultants
%\scon{B.\,B.~Baranov}
%\sconstatus{Professor}
%%\sconsnd{P.\,P.~Petrov}
%%\sconsndstatus{Professor}
%
%% City & Year
%\city{Saint Petersburg}
%\date{\number\year}
%
%\maketitle[en]

% Содержание
\tableofcontents

% Введение
\input{intro}

% Обзор литературы
%\input{review}

% Основная часть
%% Глава 1
\chapter{Название главы}
\section{Название секции}

Внутритекстовая формула $\frac{1}{\epsilon^*}=\frac{1}{\epsilon_\infty}-\frac{1}{\epsilon_0}$.
\nomenclature{$\epsilon_\infty$}{высокочастотная диэлектрическая проницаемость}
\nomenclature{$\epsilon_0$}{статическая диэлектрическая проницаемость}
Внутритекстовая формула в стиле выделенной $\dfrac{1}{\epsilon_\infty}$.
Ссылки на литературу~\cite{Yoffe_1993_AP_42_173,Efros_1982_FTP_16_7_1209,%
Anselm_1978,Segall_1968,Agranovich_1983,InP,Mishchenko_1996,Skvortsov_2008,%
Perelman_2003_math:0307245,Nielsen_2010_1006.2735,patent1,patent2}.
Ссылка на формулу~\eqref{e:Coulomb}
\begin{equation}\label{e:Coulomb}
  \frac{1}{|\vec r_1 - \vec r_2|} =
  4\pi \int \frac{d^3 q}{(2\pi)^3}\,
  \frac{e^{i\vec q(\vec r_1 - \vec r_2)}}{q^2}.
\end{equation}
где \nomencl{$\vec r_i$}{координата $i$-й частицы}.

Ссылка на рис.~\ref{f:fig}
\begin{figure}[!ht]
  \centering
  \includegraphics[width=4cm]{fig}
  \caption{\label{f:fig}%
  Подпись к рисунку.
  }
\end{figure}

\begin{wrapfigure}{r}{0.35\textwidth}
\centering
\includegraphics[width=4cm]{fig}
\caption{\label{f:ff}%
Рисунок <<в оборку>>.
}
\end{wrapfigure}

Если разность энергий электронно-дырочных уровней $E_2 - E_1$ близка к энергии продольного оптического фонона $\hbar\Omega_{\mathrm{LO}}$, то в разложении волновых функций полного гамильтониана можно ограничиться нулевым приближением для всех состояний, за исключением близких по значению к $E_2$.
Волновые функции последних представляют собой следующие комбинации вырожденных состояний\footnote{Текст сноски}.

Ссылка на таблицу~\ref{t:InPSiO2}.
\begin{table}[!ht]
  \centering
  \caption{Пример таблицы}\label{t:InPSiO2}
  \begin{tabular}{l|ccc}
    \hline\hline
    & \quad$\lambda \cdot 10^{-11}$,~$\text{дин}\cdot\text{см}^{-2}$
    & \quad$\mu \cdot 10^{-11}$,~$\text{дин}\cdot\text{см}^{-2}$
    & \quad$\rho$, $\text{г}\cdot\text{см}^{-3}$ \\
    \hline
    InP       & 3.82 & 1.69 & 4.14 \\
    SiO$_{2}$ & 1.57 & 3.11 & 2.2  \\
    \hline\hline
  \end{tabular}
\end{table}

\begin{figure}[!ht]
  \centering
  \begin{minipage}{5cm}
    \centering
    \includegraphics[width=4cm]{fig}
    \caption{Рисунок с отдельным названием}
  \end{minipage}
  \quad
  \begin{minipage}{5cm}
    \centering
    \includegraphics[width=4cm]{fig}
    \caption{Рисунок с отдельным названием}
  \end{minipage}
  \quad
  \begin{minipage}{5cm}
    \centering
    \includegraphics[width=4cm]{fig}
    \caption{Рисунок с отдельным названием}
  \end{minipage}
\end{figure}

\begin{figure}[!ht]
  \centering
  \begin{minipage}{5cm}
    \includegraphics[width=4cm]{fig}
  \end{minipage}
  \begin{minipage}{5cm}
    \includegraphics[width=4cm]{fig}
  \end{minipage}
  \caption{Рисунки с единым названием}
\end{figure}

Ссылка на внутренний рисунок (рис.~\ref{f:sub1}).

\begin{figure}[!ht]
\centering
  \begin{minipage}{5cm}
    \includegraphics[width=4cm]{fig}\subcaption{}\label{f:sub1}
  \end{minipage}
  \begin{minipage}{5cm}
    \includegraphics[width=4cm]{fig}\subcaption{}\label{f:sub2}
  \end{minipage}
  \begin{minipage}{5cm}
    \includegraphics[width=4cm]{fig}\subcaption{}\label{f:sub3}
  \end{minipage}
  \caption[]{%
  Рисунки с единым названием и подчиненной нумерацией:
    \subref{f:sub1} ссылка 1,
    \subref{f:sub2} ссылка 2,
    \subref{f:sub3} ссылка 3.
  }
\end{figure}

\subsection{Название подсекции}
Текст подсекции
\subsubsection{Название под-подсекции}
Текст под-подсекции
\paragraph{Название параграфа.}
Текст параграфа
\subparagraph{Название подпараграфа.}
Текст подпараграфа

Нумеруемый список:
\begin{enumerate}
  \item Первый уровень вложенности.
  \begin{enumerate}
    \item Второй уровень вложенности.
    \begin{enumerate}
      \item Третий уровень вложенности.
    \end{enumerate}
  \end{enumerate}
\end{enumerate}

Демонстрация полностью настраиваемых окружений типа <<теорема>>.

\newtheorem{theorem}{Теорема}[chapter]
\def\theoremstyle{}
\def\postthetheorem{:}

\newtheorem{lemm}{Лемма}[chapter]
\def\thelemmstyle{\bfseries}
\def\oparglemmstyle{}
\def\lemmstyle{}
\def\preoparglemm{(}
\def\postoparglemm{):}

\newtheorem{remark}{Примечание}[chapter]
\def\remarkstyle{\itshape}
\def\theremarkstyle{}
\def\posttheremark{:}

\begin{lemm}[Шура]
Квадратная матрица, коммутирующая со всеми матрицами неприводимого представления, кратна единичной.
\end{lemm}

\begin{theorem}
Гомоморфный образ группы изоморфен фактор-группе по ядру гомоморфизма.
\end{theorem}

\begin{remark}
Текст примечания.
\end{remark}

%% Глава 2
%\input{2}

% Заключение
\input{concl}

% Список сокращений и условных обозначений
\printnomenclature

% Словарь терминов
%\input{dict}

% Список литературы
\printbibliography[heading=bibintoc]

% Список иллюстраций
%\listoffigures

% Приложения
%\appendix
%\input{a}

\end{document}
